\documentclass{article}
\usepackage{graphicx}
\usepackage[a4paper,margin=2.5cm]{geometry}
\usepackage[T1]{fontenc}
\usepackage{amsmath, amssymb, booktabs}
\usepackage[table]{xcolor}
\usepackage{amssymb}
\usepackage{subcaption}
\usepackage{listings}
\usepackage{booktabs}
\usepackage{caption}

\usepackage[backend=biber, style=numeric]{biblatex}
\addbibresource{references.bib} % <-- Add this line
\usepackage{hyperref}

\hypersetup{
    colorlinks=true,  % false: boxed links; true: colored text
    linkcolor=black,   % color of internal links (e.g., toc)
    citecolor=blue,   % color of links to bibliography
    urlcolor=blue     % color of external links
}

\lstset{
    language=[5.0]Lua,
    basicstyle=\ttfamily\small,
    keywordstyle=\color{blue}\bfseries,
    commentstyle=\color{gray}\itshape,
    stringstyle=\color{red},
    numbers=left,
    numberstyle=\tiny\color{gray},
    stepnumber=1,
    numbersep=5pt,
    frame=single,
    breaklines=true,
    breakatwhitespace=true,
    showstringspaces=false,
    captionpos=b
}

\setlength{\parindent}{0pt}
\setlength{\parskip}{12pt}

\begin{document}

\begin{titlepage}
    \centering
    \vspace*{\fill}
    
    {\Huge\bfseries TID tests\\of NAND Flash memories\par}
    \vspace{1cm}
    {\LARGE Final report\par}
    \vspace*{\fill}
    {\Large Michał Kiedrzyński\par}
    {\normalsize TEC-EDD, ESA ESTEC\par}
    \vspace{1cm}
    {\normalsize 31 October 2025\par}
    \vspace{1cm}
    \includegraphics[width=0.4\textwidth]{images/ESA_logo.png} 

\end{titlepage}

\newpage

\tableofcontents
\newpage

\section{Introduction}
This document presents the procedure and results of Total Ionizing Dose (TID) tests of 32Gbit NAND Flash memories from Micron (MT29F32G08ABAAAWP-ITZ:A). The model was chosen as it is expected to be used in the PW-Sat3 satellite, developed by Students' Space Association at Warsaw University of Technology.

The tests were conducted from 20.10.2025 to 31.10.2025 in the Co-60 facility at the European Space Agency's (ESA) European Space Research and Technology Centre (ESTEC) in Noordwijk, Netherlands.

\section{Facility}
The Co-60 facility at ESTEC is an ISO17025-accredited gamma radiation testing and calibration laboratory. It uses Cobalt-60 (Co-60) as the radiation source, which is a radioactive isotope with half-life of 5.25 years. Co-60 decays by beta decay to an excited state of Nickel-60 (Ni-60), which then emits two gamma rays with energies of approximately 1.173 MeV and 1.332 MeV. \parencite{escies}

The source consists of multiple small rods about 50 mm long, held around the periphery of a 30mm diameter container. When the source is raised to the irradiation position, the gamma beam produced by the Co-60 decay exits the irradiator unit through a collimator window into the radiation cell. The source schematic can be found in figure \ref{fig:source_schematic}. \parencite{escies}

\begin{figure}[h]
    \centering
    \includegraphics[width=0.9\textwidth]{images/source_schematic.png}
    \caption{Source schematic}
    \label{fig:source_schematic}
\end{figure}

The facility consists of the radiation cell a and large external control room with 14 cable feed-throughs that enable the remote monitoring and controlling of experiments. Additionally, it features a Remote Positioning System, consisting of a trolley on a rail, for easy dose rate manipulation. The facility layout can be found in figure \ref{fig:facility_plan} \parencite{escies}, and photographs in figure \ref{fig:irradiation_setup}.


\begin{figure}[h]
    \centering
    \includegraphics[width=0.9\textwidth]{images/facility_schematic.jpg}
    \caption{Co-60 facility layout}
    \label{fig:facility_plan}
\end{figure}

\begin{figure}[h]
    \hfill
    \begin{subfigure}[b]{0.45\textwidth}
        \includegraphics[width=1\textwidth]{images/irr0.jpg}
        \caption{Devices prepared for irradiation}
    \end{subfigure}
    \hfill
    \begin{subfigure}[b]{0.45\textwidth}
        \includegraphics[width=1\textwidth]{images/irr1.jpg}
        \caption{Irradiation chamber}
    \end{subfigure}
    \hfill
    \caption{Irradiation setup}
    \label{fig:irradiation_setup}
\end{figure}

\section{Setup}
\subsection{Devices Under Test (DUTs)}
The test device was Micron's MT29F32G08ABAAAWP-ITZ:A 32Gbit NAND Flash memory die. It consists of 4096 blocks of 128 pages, with each page containing 8640 bytes (8192 bytes data + 448 bytes spare). The memory uses Open NAND Flash Interface (ONFI) 2.2 for communication and comes in a 48-pin TSOP package. The key parameters of the memory are listed in table \ref{tab:memory_params}. For convenient access to the memory pins, the devices were soldered onto dedicated TSOP-48 to DIP-48 adapters, as depicted in figure \ref{fig:soldered}.

\begin{table}[h]
    \centering
    \caption{Key parameters of MT29F32G08ABAAAWP-ITZ:A memory}
    \renewcommand{\arraystretch}{1.2} 
    \begin{tabular}{@{}ll@{}}
        \toprule
        \textbf{Parameter} &  \textbf{Value} \\
        \midrule
        Memory size & 32 Gbit (4 GB) \\
        Page size & 8192 bytes + 448 bytes spare \\
        Block size & 128 pages (1 MB) \\
        Number of blocks & 4096 \\
        Interface & ONFI 2.2 \\
        Package type & 48-pin TSOP \\
        Operating voltage & 2.7V - 3.6V \\
        Endurance & Up to 80,000 program/erase cycles \\
        Operating temperature & -40°C to +85°C \\
        Device width & 8 bits \\
        \bottomrule
    \end{tabular}
    \label{tab:memory_params}
\end{table}

\begin{figure}
    \centering
    \includegraphics[width=0.7\textwidth]{images/soldered.jpg}
    \caption{DUTs soldered onto TSOP-48 to DIP-48 adapters}
    \label{fig:soldered}
\end{figure}

A total of 6 devices were tested - 3 were biased, 2 unbiased, and 1 was used as a control sample (not irradiated). The DUTs overview is presented in table \ref{tab:devices}.
\begin{table}[h]
    \centering
    \caption{Devices Under Test (DUTs) overview}
    \renewcommand{\arraystretch}{1.2} 
    \begin{tabular}{@{}lll@{}}
        \toprule
        \textbf{Device ID} &  \textbf{Bias Condition} & \textbf{Purpose} \\
        \midrule
        DUT1 & Biased   & Test sample \\
        DUT2 & Biased   & Test sample \\
        DUT3 & Unbiased   & Test sample \\
        DUT4 & Biased & Test sample \\
        DUT5 & Unbiased & Test sample \\
        CTRL & N/A      & Control sample \\
        \bottomrule
    \end{tabular}
    \label{tab:devices}
\end{table}

\subsection{Measurement setup}
\subsubsection{Custom FPGA controller}
\subsection{Irradiation setup}
\subsection{Annealing setup}

\section{Procedure}
\subsection{Test plan}
\subsection{Test procedure}

\section{Results}
<pure data>

\section{Discussion}
\subsection{Summary}
\subsection{Biased vs Unbiased}
\subsection{Annealing effects}

\printbibliography

\end{document}