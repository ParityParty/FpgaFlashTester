\documentclass{article}
\usepackage{graphicx}
\usepackage[a4paper,margin=2.5cm]{geometry}
\usepackage[T1]{fontenc}
\usepackage{amsmath, amssymb, booktabs}
\usepackage[table]{xcolor}
\usepackage{amssymb}
\usepackage{subcaption}
\usepackage{listings}
\usepackage{booktabs}
\usepackage{caption}

\usepackage[backend=biber, style=numeric]{biblatex}
\addbibresource{references.bib} % <-- Add this line
\usepackage{hyperref}

\hypersetup{
    colorlinks=true,  % false: boxed links; true: colored text
    linkcolor=black,   % color of internal links (e.g., toc)
    citecolor=blue,   % color of links to bibliography
    urlcolor=blue     % color of external links
}

\lstset{
    language=[5.0]Lua,
    basicstyle=\ttfamily\small,
    keywordstyle=\color{blue}\bfseries,
    commentstyle=\color{gray}\itshape,
    stringstyle=\color{red},
    numbers=left,
    numberstyle=\tiny\color{gray},
    stepnumber=1,
    numbersep=5pt,
    frame=single,
    breaklines=true,
    breakatwhitespace=true,
    showstringspaces=false,
    captionpos=b
}

\setlength{\parindent}{0pt}
\setlength{\parskip}{12pt}

\begin{document}

\begin{titlepage}
    \centering
    \vspace*{\fill}
    
    {\Huge\bfseries TID tests\\of NAND Flash memories\par}
    \vspace{1cm}
    {\LARGE Final report\par}
    \vspace*{\fill}
    {\Large Michał Kiedrzyński\par}
    {\normalsize TEC-EDD, ESA ESTEC\par}
    \vspace{1cm}
    {\normalsize 31 October 2025\par}
    \vspace{1cm}
    \includegraphics[width=0.4\textwidth]{images/ESA_logo.png} 

\end{titlepage}

\newpage

\tableofcontents
\newpage

\section{Introduction}
This document presents the procedure and results of Total Ionizing Dose (TID) tests of 32Gbit NAND Flash memories from Micron (MT29F32G08ABAAAWP-ITZ:A). The model was chosen as it is expected to be used in the PW-Sat3 satellite, developed by Students' Space Association at Warsaw University of Technology.

The tests were conducted from 20.10.2025 to 31.10.2025 in the Co-60 facility at the European Space Agency's (ESA) European Space Research and Technology Centre (ESTEC) in Noordwijk, Netherlands.

\section{Facility}
The Co-60 facility at ESTEC is an ISO17025-accredited gamma radiation testing and calibration laboratory. It uses Cobalt-60 (Co-60) as the radiation source, which is a radioactive isotope with half-life of 5.25 years. Co-60 decays by beta decay to an excited state of Nickel-60 (Ni-60), which then emits two gamma rays with energies of approximately 1.173 MeV and 1.332 MeV. \parencite{escies}

The source consists of multiple small rods about 50 mm long, held around the periphery of a 30mm diameter container. When the source is raised to the irradiation position, the gamma beam produced by the Co-60 decay exits the irradiator unit through a collimator window into the radiation cell. The source schematic can be found in figure \ref{fig:source_schematic}. \parencite{escies}

\begin{figure}[h]
    \centering
    \includegraphics[width=0.9\textwidth]{images/source_schematic.png}
    \caption{Source schematic}
    \label{fig:source_schematic}
\end{figure}

The facility consists of the radiation cell and a large external control room with 14 cable feed-throughs that enable the remote monitoring and controlling of experiments. Additionally, it features a Remote Positioning System, consisting of a trolley on a rail, for easy dose rate manipulation. The facility layout can be found in figure \ref{fig:facility_plan} \parencite{escies}, and photographs in figure \ref{fig:irradiation_setup}.


\begin{figure}[h]
    \centering
    \includegraphics[width=0.9\textwidth]{images/facility_schematic.jpg}
    \caption{Co-60 facility layout}
    \label{fig:facility_plan}
\end{figure}

\section{Setup}
\subsection{Devices Under Test (DUTs)}
The test device was Micron's MT29F32G08ABAAAWP-ITZ:A 32Gbit NAND Flash memory die. It consists of 4096 blocks of 128 pages, with each page containing 8640 bytes (8192 bytes data + 448 bytes spare). The memory uses Open NAND Flash Interface (ONFI) 2.2 for communication and comes in a 48-pin TSOP package. The key parameters of the memory are listed in table \ref{tab:memory_params}. For convenient access to the memory pins, the devices were soldered onto dedicated TSOP-48 to DIP-48 adapters, as depicted in figure \ref{fig:soldered}.

\begin{table}[h]
    \centering
    \caption{Key parameters of MT29F32G08ABAAAWP-ITZ:A memory}
    \renewcommand{\arraystretch}{1.2} 
    \begin{tabular}{@{}ll@{}}
        \toprule
        \textbf{Parameter} &  \textbf{Value} \\
        \midrule
        Memory size & 32 Gbit (4 GB) \\
        Page size & 8192 bytes + 448 bytes spare \\
        Block size & 128 pages (1 MB) \\
        Number of blocks & 4096 \\
        Interface & ONFI 2.2 \\
        Package type & 48-pin TSOP \\
        Operating voltage & 2.7V - 3.6V \\
        Endurance & Up to 80,000 program/erase cycles \\
        Operating temperature & -40°C to +85°C \\
        Device width & 8 bits \\
        \bottomrule
    \end{tabular}
    \label{tab:memory_params}
\end{table}

\begin{figure}
    \centering
    \includegraphics[width=0.7\textwidth]{images/soldered.jpg}
    \caption{DUTs soldered onto TSOP-48 to DIP-48 adapters}
    \label{fig:soldered}
\end{figure}

A total of 6 devices were tested - 3 were biased, 2 unbiased, and 1 was used as a control sample (not irradiated). The DUTs overview is presented in table \ref{tab:devices}.
\begin{table}[h]
    \centering
    \caption{Devices Under Test (DUTs) overview}
    \renewcommand{\arraystretch}{1.2} 
    \begin{tabular}{@{}lll@{}}
        \toprule
        \textbf{Device ID} &  \textbf{Bias Condition} & \textbf{Purpose} \\
        \midrule
        DUT1 & Biased   & Test sample \\
        DUT2 & Biased   & Test sample \\
        DUT3 & Unbiased   & Test sample \\
        DUT4 & Biased & Test sample \\
        DUT5 & Unbiased & Test sample \\
        CTRL & N/A      & Control sample \\
        \bottomrule
    \end{tabular}
    \label{tab:devices}
\end{table}

\subsection{Irradiation setup}
For irradiation, the DUTs were placed on a breakout board and connected to a 3-channel power supply. The connection diagram is presented in figure \ref{fig:irr_diagram}. For biased devices, voltage of 3.3V was applied to $V_{cc}$ pins, $V_{ss}$ pins were grounded, and the rest were left floating. Unbiased devices had all their pins grounded. No reading or writing operations were performed during irradiation.

Upon making appropriate connections and checking the power lines with a multimeter, the breakout board was carefully positioned inside the irradiation chamber using three laser levels to ensure correct alignment. Tape measures and rulers were additionally used to verify the distance from the source.

A dosimeter was placed alongside the DUTs to monitor the received dose in real-time and keep an accurate record of the accumulated dose. The chamber was then closed, and the irradiation process commenced according to the test plan. Photographs depicting the irradiation setup can be found in figure \ref{fig:irradiation_setup}.

\begin{figure}[h]
    \centering
    \includegraphics[width=1\textwidth]{images/irr_diagram.drawio.png}
    \caption{Irradiation setup diagram}
    \label{fig:irr_diagram}
\end{figure}

\begin{figure}[h]
    \hfill
    \begin{subfigure}[b]{0.45\textwidth}
        \includegraphics[width=1\textwidth]{images/irr0.jpg}
        \caption{Devices prepared for irradiation}
    \end{subfigure}
    \hfill
    \begin{subfigure}[b]{0.45\textwidth}
        \includegraphics[width=1\textwidth]{images/irr1.jpg}
        \caption{Irradiation chamber}
    \end{subfigure}
    \hfill
    \caption{Irradiation setup}
    \label{fig:irradiation_setup}
\end{figure}

\subsection{Measurement setup}
\subsubsection{Custom FPGA controller}
Since Micron's NAND Flash memories use the well-documented ONFI interface and budget constraints were a consideration, a custom FPGA-based controller was designed to facilitate reading and writing operations. The design was implemented on an already available, UltraScale+ -based Kria KR260 Robotics Starter Kit using VHDL.

The block diagram of the controller is presented in figure \ref{fig:fpga_design}. The architecture consists of four modules:
\begin{itemize}
    \item \textbf{NAND Controller} - Performs low-level operations such as page read, page program, and block erase, acting directly on the interface signals.
    \item \textbf{Flash Programmer} - Contains the higher-level test logic to execute the test plan, including sequences of read/write/erase operations and error checking.
    \item \textbf{UART TX} - facilitates serial communication between the FPGA and a host PC for sending the status updates (capable only of transmitting data).
    \item \textbf{Clocking Wizard} - Generates the required clock frequency for the design and provides the reset signal from one of the oscillators present on the KR260 board.
\end{itemize}

\begin{figure}[h]
    \centering
    \includegraphics[width=1\textwidth]{images/fpga_design.png}
    \caption{FPGA design for NAND flash tests}
    \label{fig:fpga_design}
\end{figure}

\subsubsection{Measurement station}
The measurement station consisted of the KR260 board with the custom FPGA controller, a host PC, a power supply, and a breakout board for easy connection to the DUTs. The setup diagram is depicted in figure \ref{fig:measurement_diagram} and the photographs in figure \ref{fig:measurement_setup}.

\begin{figure}[h]
    \centering
    \includegraphics[width=0.8\textwidth]{images/meas_diagram.drawio.png}
    \caption{Measurement station setup diagram}
    \label{fig:measurement_diagram}
\end{figure}

\begin{figure}[h]
    \hfill
    \begin{subfigure}[b]{0.45\textwidth}
        \includegraphics[width=1\textwidth]{images/setup0.jpg}
        \caption{Memory die connected to the FPGA}
    \end{subfigure}
    \hfill
    \begin{subfigure}[b]{0.45\textwidth}
        \includegraphics[width=1\textwidth]{images/setup1.jpg}
        \caption{Measurement station}
    \end{subfigure}
    \hfill
    \caption{Measurement setup}
    \label{fig:measurement_setup}
\end{figure}

\subsection{Annealing setup}

\section{Procedure}
\subsection{Test plan}
\subsection{Test procedure}

\section{Results}
<pure data>

\section{Discussion}
\subsection{Summary}
\subsection{Biased vs Unbiased}
\subsection{Annealing effects}

\printbibliography

\end{document}